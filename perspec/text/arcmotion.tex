\documentclass[12pt]{article}

\usepackage{amsmath}

\begin{document}

\noindent Consider a robot moving by distance $\Delta s$ in an arc of a
circle of radius $R$.  To be precise, $\Delta s$ is the arc length
moved by a point between the differentially rotating wheels.

The camera focal point is forward by a known distance $c$ from that
point.  The camera is pointing forward (for now) and its height is
known but not important (for now).  The camera focal point is the
origin of coordinates: $x$ is to the right, $y$ is down, and $z$ is
forward.  In the camera plane $X$ is to the right and $Y$ is down.

Let $\phi=\Delta s/R$.  The motion of the camera then amounts to
moving back by $c$, left by $R$, rotating by $\phi$, and then
reversing the translations in the new (rotated) coordinate system.  A
point at $x,y,z$ will have new coordinates $x',y',z'$ given by
\begin{equation}
  \begin{pmatrix} x' + R \\ y' \\ z' + c \end{pmatrix} =
  \begin{pmatrix}
    \cos\phi & 0 & \sin\phi \\
    0        & 1 & 0 \\
   -\sin\phi & 0 & \cos\phi
  \end{pmatrix}
  \begin{pmatrix} x  + R \\ y \\ z  + c \end{pmatrix}
\end{equation}
We now introduce the visual vectors
\begin{equation}
\begin{aligned}
  \Delta X &= \frac{x'}{z'}-\frac xz \\
  \Delta Y &= \frac{y'}{z'}-\frac yz
\end{aligned}
\end{equation}
Sorting through the algebra (a computer helps here) we get
\begin{equation}\label{arrows}
\begin{aligned}
  \Delta X &= \frac Xz \Delta s + (X^2+Y^2 + c/z) \frac{\Delta s}R
  + O\left(\Delta s^2/z\right) \\
  \Delta Y &= \frac Yz \Delta s + XY \,\frac{\Delta s}R
  + O\left(\Delta s^2/z\right)
\end{aligned}
\end{equation}
The first terms on the right are the visual vectors radiating out from
the vanishing point.  The following terms with $\Delta s/R$ are
perturbations of these radial visual vectors.

If $\Delta s$ and $R$ are known from wheel odometry, and their ratio
is small, then the pixel values $X,Y$ and visual vectors
$\Delta X,\Delta Y$
through eqn.~\eqref{arrows} provide $x,y$ and $z$ with redundancy.

As mentioned above, we have assumed for now that the camera is
calibrated and pointing forward.

\end{document}
